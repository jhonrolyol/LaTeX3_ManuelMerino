
% Preamble
    \documentclass[aspectratio = 169, 11pt ]{beamer}


% Packages
    \usepackage[utf8]{inputenc}
    \usepackage[spanish]{babel}
    \usepackage[T1]{fontenc}
    \usepackage{graphicx}
    \usepackage{amsfonts}
    \usepackage{amsmath}
    \usepackage{amssymb}
    \usepackage{lipsum}

% Tikz
    \usepackage{tikz}
    

% Theme
    \usetheme{Madrid}

% Modificamos el color
    \usepackage{xcolor}
    \definecolor{azul}{rgb}{0.17, 0.40, 0.69}

    \setbeamercolor{structure}{fg = azul} % Se modifica el color de la estructura

% Justificación
    \usepackage{ragged2e}

% Title
    \title{Mi primer documento en Beamer con \LaTeX}
    \subtitle{Sesión 6}
    \author[Manuel Merino]{Manuel Merino Huaman}
    \date{\today} 
    \institute[UNMSM]{
        Bach. Computación Científica - UNMSM
    }
    
    
% Document
    \begin{document}
        % Title
            {
            \usebackgroundtemplate{\includegraphics{06_PRESENTACIÓN_Y_PÓSTER_CON_BEAMER/Portada.pdf}}            \begin{frame}[plain]
                   \begin{center}
                       {\Large \textsc{\textrm{Universidad Nacional Mayor de San Marcos}}}\\
                       {\textrm{Universidad del Perú. Decana de América}}\\
                       \vspace{1cm}
                       \textbf{\textsc{\textrm{\inserttitle}}}\\ % Para usar en negrita ([T1]{fontenc})
                       \insertsubtitle
                   \end{center} 
              \end{frame}
            }

        % Table of contents
            \begin{frame}{Contenido}
                \tableofcontents
            \end{frame}
        
        % Section 1
            \section{Introducción}
                \begin{frame}{Introducción}
                    \justifying
                    \lipsum[8]
                \end{frame}

        % Section 2
            \section{Objetivos}
                \begin{frame}{Objetivos}
                        \begin{figure}[ht]
                            \centering
                            \begin{tikzpicture}[scale = 1.6, samples = 70]
                            \fill[red] (0,0) -- plot[domain = 0:pi] ({\x}, {sin(\x r)}) -- (pi,0) -- cycle;
                            \draw[dashed, xstep = pi/2] (-pi/4, -1) grid (9*pi/4, 1);
                            \draw[<-] (7,0) node[below right] {$x$} -- (-0.5,0); 
                            \draw[<-] (0,1.25) node[above left] {$y$} -- (0, -1.25); 
                            \node[below right] at (pi/2, 0) {$\frac{\pi}{2}$};
                            \node[below right] at (pi, 0) {$\pi$};
                            \node[below right] at (3*pi/2, 0) {$\frac{3\pi}{2}$};
                            \node[below right] at (2*pi, 0) {$2\pi$};
                            \node[left] at (0,1) {$1$};
                            \node[left] at (0,-1) {$-1$};
                            \draw[very thick, azul, domain = -0.5 + 0:2*pi + 0.5] plot({\x}, {sin(\x r)});
                        \end{tikzpicture}
                        \caption{Función seno}
                        \label{fig: Tikz7}
                    \end{figure}
                \end{frame}
                
        % Section 3
            \section{Metodología}
                \begin{frame}{Metodología}
                    Hola
                \end{frame}
            
        % Section 4
            \section{Análisis y resultados}
                \begin{frame}{Análisis y resultados}
                    Hola
                \end{frame}

        % Section 5
            \section{Conclusiones}
                \begin{frame}{Conclusiones}
                    Hola
                \end{frame}
        
    \end{document}






