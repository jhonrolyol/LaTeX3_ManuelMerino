% PREÁMBULO -------------------------------------

% Definir la clase de documento
\documentclass[a4paper, 11pt]{article}

% Codificación de entrada
\usepackage[utf8]{inputenc}

% Idioma
\usepackage[spanish]{babel}

% Geometría
\usepackage[left = 2cm, right = 2.5cm, top = 3cm, bottom = 2.2cm]{geometry}

% Inserción de figuras
\usepackage{graphicx}

% Hipervínculos
\usepackage{hyperref}

\usepackage{titlesec}

% Matemáticas
\usepackage{amsfonts}
\usepackage{amsmath}
\usepackage{amssymb}

% Crear párrafos
\usepackage{lipsum}


% Multi columnas
\usepackage{multicol}

%Colores
\usepackage{xcolor}
\definecolor{gris}{gray}{0.9}
\definecolor{azul}{rgb}{0.0, 0.48, 0.65}

% Información del documento
\author{Manuel Merino Huaman}
\title{Primera Sesión de \LaTeX}
\date{26 de febrero de 2023}

% ----------------------------------------------------


\begin{document}

    \maketitle
    
    \begin{abstract}
        \lipsum[9]\\
        
        \noindent
        \textbf{Palabras clave:} Machine Learning, Probabilidad, Espacio del Hilbert, Cáncer.
    \end{abstract}
    
    
    
    
    \begin{multicols}{2}
        
        \section{Introducción}\label{sec: intro}
            \lipsum[3] \\
            
            \lipsum[9]\\
            
            \lipsum[1]
        
        \section{Metodología}\label{sec: met}
            \lipsum[2]\\
            
            \lipsum[1] \lipsum[9]\\
            
            \lipsum[5]
            
            
        
        \section{Bases teóricas}\label{sec: base teo}
            \lipsum[9]
            \subsection{Machine learning}\label{subsec: ML}
                \lipsum[11]
                
                \subsubsection{Modelos clasificación}\label{subsubsec: Clasi}
                    \lipsum[1] \lipsum[4]\\
                    
                    \lipsum[2]\\
                    
                    \lipsum[5] \lipsum[1]
                \subsubsection{Modelos de regresión}\label{subsubsec: Reg}
                    \lipsum[9]\\
                    
                    \lipsum[7] \lipsum[2]
                    
            \subsection{Cáncer de pulmón}\label{subsec: Cancer}
                \lipsum[1]
            
        \section{Análisis y Resultados}\label{sec: A y R}
            \lipsum \\
            
            Como se mencionó en la sección \ref{subsubsec: Clasi}, podemos decir 
    
        \section{Conclusión}\label{sec: Conclu}
            \lipsum[9] \lipsum[8]
    \end{multicols}
    
    
    
    
    
        

\end{document}
