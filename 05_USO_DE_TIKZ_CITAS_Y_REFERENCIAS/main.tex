% PREÁMBULO -------------------------------------
  
% Definir la clase de documento
\documentclass[a4paper, 11pt]{article}

% Codificación de entrada
\usepackage[utf8]{inputenc}

% Idioma
\usepackage[spanish]{babel}

% Geometría
\usepackage[left = 2cm, right = 2.5cm, top = 3cm, bottom = 2.2cm]{geometry}

% Inserción de figuras
\usepackage{graphicx}
%\usepackage{subcaption}

% Hipervínculos
\usepackage{hyperref}

\usepackage{titlesec}

% Matemáticas
\usepackage{amsfonts}
\usepackage{amsmath}
\usepackage{amssymb}
\usepackage{amsthm}
\newtheorem{defi}{Definición}[section]
\newtheorem{lema}{Lema}[section]
\newtheorem{teo}{Teorema}[section]
\newtheorem{coro}{Corolario}[section]


\usepackage{float}

% Crear párrafos
\usepackage{lipsum}

% Multicolumnas
\usepackage{multicol}


% Espacios en párrafos
\usepackage{setspace}




%Colores
\usepackage{xcolor}
\definecolor{gris}{gray}{0.9}
\definecolor{azul}{rgb}{0.0, 0.48, 0.65}

% Información del documento
\author{Manuel Merino Huaman}
\title{Modo matemático en \LaTeX}
\date{26 de febrero de 2023}

% Usar el paquete tikz
\usepackage{tikz}

% Configuraciones en los listados de texto - Numerados
%% Nivel 1
\renewcommand{\theenumi}{\arabic{enumi}}
\renewcommand{\labelenumi}{\arabic{enumi}.}

%% Nivel 2
\renewcommand{\theenumii}{\alph{enumii}}
\renewcommand{\labelenumii}{\arabic{enumii}.$\alph{enumii}$}

%% Nivel 3
\renewcommand{\theenumiii}{\roman{enumiii}}
\renewcommand{\labelenumiii}{\roman{enumiii})}

%% Nivel 4
\renewcommand{\theenumiv}{\alph{enumiv}}
\renewcommand{\labelenumiv}{\alph{enumiv}.}

% Configuraciones en los listados de texto - No numerados
%% Nivel 1
\renewcommand{\labelitemi}{$\checkmark$}

%% Nivel 2
\renewcommand{\labelitemii}{$\bigstar$}

%% Nivel 3
\renewcommand{\labelitemiii}{$\dagger$}

%% Nivel 4
\renewcommand{\labelitemiv}{$\circledast$}

% Creación de comandos
%% Números naturales
\newcommand{\N}{\mathbb{N}}
%% Números enteros
\newcommand{\Z}{\mathbb{Z}}
%% Números racionales
\newcommand{\Q}{\mathbb{Q}}
%% Números irracionales
\newcommand{\I}{\mathbb{I}}
%% Números reales
\newcommand{\R}{\mathbb{R}}
%% Números complejos
\newcommand{\C}{\mathbb{C}}

% Crear funciones trigonométricas inversas
\newcommand{\arccot}{\text{arccot}}
\newcommand{\arcsec}{\text{arcsec}}
\newcommand{\arccsc}{\text{arccsc}}


\usepackage{rotating}
\usepackage{tabularx} % Permite hacer cambios en el contenido de una tabla
\usepackage{multirow}

\usepackage{caption}
%\captionsetup[table]{
%   name = Tabla,
%   labelsep = newline,
%   justification = raggedright,
%   singlelinecheck = false,
%   labelfont = bf
%   }

 
% Máximo entero
\usepackage{mathabx}


% Entorno del documento
\begin{document}

    \maketitle

    % Figure 1
    \begin{figure}[ht]
        \centering
        \begin{tikzpicture}
            \draw (0,0) -- (1,4) -- (3,0.5) -- (4,1);
        \end{tikzpicture}
        \caption{Mi primera imagen con Tikz}
        \label{fig: Tikz1}
    \end{figure}
    
    % Figure 2
    \begin{figure}[ht]
        \centering
        \begin{tikzpicture}[xscale = 2, yscale = 0.5]
            \draw[<-] (6,0) -- (0,0);
            \draw[<-] (0,6) -- (0,0);
        \end{tikzpicture}
        \caption{Mi segunda imagen con Tikz}
        \label{fig: Tikz2}
    \end{figure}

    % Figure 3
    \begin{figure}[ht]
        \centering
        \begin{tikzpicture}
            \draw[<-,dashed, ultra thick] (0,3) -- (5,3);
            \draw[dotted, very thick] (0,2) -- (5,2);
            \draw[red, thick] (0,1) -- (5,1);
            \draw[dashed, semithick] (0,0) -- (5,0);
            \draw[azul, dashed, thin] (0,-1) -- (5,-1);
            \draw[very thin] (0,-2) -- (5,-2);
            \draw[ultra thin] (0,-3) -- (5,-3);
        \end{tikzpicture}
        \caption{Mi tercera imagen con Tikz}
        \label{fig: Tikz3}
    \end{figure}


    % Figure 4
    \begin{figure}[ht]
        \centering
        \begin{tikzpicture}
            \draw[red, dashed, very thick, fill = lightgray] (1,1) circle[radius = 2];
            \draw[fill = black] (1,1) circle[radius = 0.05];
            \node[below right] at (1,1) {$(1,1)$};
            \draw (-3,-4) rectangle (1,-2);
            \draw[<-] (1,1) -- (1,3) -- (1,3) -- (1,3); 
            \draw (6,0) -- (-6,0);
            \draw (0,6) -- (0,-6);
        \end{tikzpicture}
        \caption{Mi cuarta imagen con Tikz}
        \label{fig: Tikz4}
    \end{figure}


    % Figure 5
    \begin{figure}[ht]
       \centering
       \begin{tikzpicture}
            \draw[red, dashed, very thick, fill = lightgray] (1, 1) circle[radius = 2];
            \draw[fill = black] (1, 1) circle[radius = 0.05];
            \node[below right] at (1, 1) {$(1, 1)$};
            \draw (-3, -4) rectangle (1, -2);
            \draw[darkgray, dashed] (-6, 1) -- (6, 1);
            \draw[darkgray, dashed] (-6, 2) -- (6, 2);
            \draw[darkgray, dashed] (-6, 3) -- (6, 3);
            \draw[darkgray, dashed] (-6, 4) -- (6, 4);
            \draw[darkgray, dashed] (-6, 5) -- (6, 5);
            \draw[darkgray, dashed] (1, -5) -- (1, 5);
            \draw[darkgray, dashed] (2, -5) -- (2, 5);
            \draw[darkgray, dashed] (3, -5) -- (3, 5);
            \draw[darkgray, dashed] (4, -5) -- (4, 5);
            \draw[darkgray, dashed] (5, -5) -- (5, 5);
            \draw[darkgray, dashed] (-1, -5) -- (-1, 5);
            \draw[darkgray, dashed] (-2, -5) -- (-2, 5);
            \draw[darkgray, dashed] (-3, -5) -- (-3, 5);
            \draw[darkgray, dashed] (-4, -5) -- (-4, 5);
            \draw[darkgray, dashed] (-5, -5) -- (-5, 5);
            \draw[<-] (6, 0) -- (-6, 0);
            \draw[<-] (0, 6) -- (0, -6);
       \end{tikzpicture}
       \caption{Plano cartesiano}
       \label{fig: tikz 5}
   \end{figure}

   % Figure 6
    \begin{figure}[ht]
        \centering
        \begin{tikzpicture}[scale = 0.26]
            \draw[lightgray, fill = gris] (-6,-6) rectangle (6,6);
            \draw[dashed, white ] (-6,-6) grid (6,6);
            
            \draw[<-] (6,0) node[below right] {$x$} (-6,0);
            \draw[<-] (0,6) node[below left] {$y$} (0,-6);

            \draw[azul, domain = -3:1] plot ({\x}, {pow(\x,2) - 3});
            \draw[red, domain = 1:5] plot({\x}, {sqrt(\x) - 3});

            \draw[black, domain = -5:1.7] plot({\x}, {exp(\x)});
            \draw[black, domain = -5:1.7] plot({\x}, {-exp(\x)});
            
        \end{tikzpicture}
        \caption{Plano cartesiano 2}
        \label{fig: Tikz6}
    \end{figure}


    % Figura 7
    \begin{figure}[ht]
        \centering
        \begin{tikzpicture}[scale = 2, samples = 70]
            \fill[red] (0,0) -- plot[domain = 0:pi] ({\x}, {sin(\x r)}) -- (pi,0) -- cycle;
            \draw[dashed, xstep = pi/2] (-pi/4, -1) grid (9*pi/4, 1);
        
            \draw[<-] (7,0) node[below right] {$x$} -- (-0.5,0); 
            \draw[<-] (0,1.25) node[above left] {$y$} -- (0, -1.25); 

            \node[below right] at (pi/2, 0) {$\frac{\pi}{2}$};
            \node[below right] at (pi, 0) {$\pi$};
            \node[below right] at (3*pi/2, 0) {$\frac{3\pi}{2}$};
            \node[below right] at (2*pi, 0) {$2\pi$};

            \node[left] at (0,1) {$1$};
            \node[left] at (0,-1) {$-1$};
     
            \draw[very thick, azul, domain = -0.5 + 0:2*pi + 0.5] plot({\x}, {sin(\x r)});
            
        \end{tikzpicture}
        \caption{Función seno}
        \label{fig: Tikz7}
    \end{figure}

    % Figura 7
    \begin{figure}[ht]
        \centering
        \begin{tikzpicture}
            \draw[<-] (5,0) node[below right] {$x$} -- (-3,0);
            \draw[<-] (0,8) -- (0,-8);

            \draw[azul, domain = -pi/2 + 0.15:pi/2 -0.15] plot({\x}, {tan(\x r)});
            \draw[azul, domain = pi/2 + 0.125:3*pi/2 -0.125] plot({\x}, {tan(\x r)});

            \draw[red, dashed, very thick] (-pi/2,8) -- (-pi/2,-8);
            \draw[red, dashed, very thick] (pi/2,8) -- (pi/2,-8);
            \draw[red, dashed, very thick] (3*pi/2,8) -- (3*pi/2,-8);

            
        \end{tikzpicture}
        \caption{Caption}
        \label{fig:my_label}
    \end{figure}
    
    
\end{document}










