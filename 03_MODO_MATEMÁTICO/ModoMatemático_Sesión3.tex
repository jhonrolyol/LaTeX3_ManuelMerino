% PREÁMBULO -------------------------------------
  
% Definir la clase de documento
\documentclass[a4paper, 11pt]{article}

% Codificación de entrada
\usepackage[utf8]{inputenc}

% Idioma
\usepackage[spanish]{babel}

% Geometría
\usepackage[left = 2cm, right = 2.5cm, top = 3cm, bottom = 2.2cm]{geometry}

% Inserción de figuras
\usepackage{graphicx}

% Hipervínculos
\usepackage{hyperref}

\usepackage{titlesec}

% Matemáticas
\usepackage{amsfonts}
\usepackage{amsmath}
\usepackage{amssymb}
\usepackage{amsthm}
\newtheorem{defi}{Definición}[section]
\newtheorem{lema}{Lema}[section]
\newtheorem{teo}{Teorema}[section]
\newtheorem{coro}{Corolario}[section]




% Crear párrafos
\usepackage{lipsum}

% Multicolumnas
\usepackage{multicol}


% Espacios en párrafos
\usepackage{setspace}




%Colores
\usepackage{xcolor}
\definecolor{gris}{gray}{0.9}
\definecolor{azul}{rgb}{0.0, 0.48, 0.65}

% Información del documento
\author{Manuel Merino Huaman}
\title{Modo matemático en \LaTeX}
\date{26 de febrero de 2023}

% Configuraciones en los listados de texto - Numerados
%% Nivel 1
\renewcommand{\theenumi}{\arabic{enumi}}
\renewcommand{\labelenumi}{\arabic{enumi}.}

%% Nivel 2
\renewcommand{\theenumii}{\alph{enumii}}
\renewcommand{\labelenumii}{\arabic{enumii}.$\alph{enumii}$}

%% Nivel 3
\renewcommand{\theenumiii}{\roman{enumiii}}
\renewcommand{\labelenumiii}{\roman{enumiii})}

%% Nivel 4
\renewcommand{\theenumiv}{\alph{enumiv}}
\renewcommand{\labelenumiv}{\alph{enumiv}.}

% Configuraciones en los listados de texto - No numerados
%% Nivel 1
\renewcommand{\labelitemi}{$\checkmark$}

%% Nivel 2
\renewcommand{\labelitemii}{$\bigstar$}

%% Nivel 3
\renewcommand{\labelitemiii}{$\dagger$}

%% Nivel 4
\renewcommand{\labelitemiv}{$\circledast$}

% Creación de comandos
%% Números naturales
\newcommand{\N}{\mathbb{N}}
%% Números enteros
\newcommand{\Z}{\mathbb{Z}}
%% Números racionales
\newcommand{\Q}{\mathbb{Q}}
%% Números irracionales
\newcommand{\I}{\mathbb{I}}
%% Números reales
\newcommand{\R}{\mathbb{R}}
%% Números complejos
\newcommand{\C}{\mathbb{C}}

% Crear funciones trigonométricas inversas
\newcommand{\arccot}{\text{arccot}}
\newcommand{\arcsec}{\text{arcsec}}
\newcommand{\arccsc}{\text{arccsc}}

% Máximo entero
\usepackage{mathabx}



% Entorno del documento
\begin{document}

\maketitle

\section{Introducción}
    Si yo deseo escribir una fracción en un párrafo, hago uso del comando \verb|\frac{num}{den}| en modo matemático, en ,donde se obtiene $\frac{1}{2}$, donde \verb|num| es el numerador y \verb|den| es el denominador, por ejemplo si escribimos \verb|$$$\frac{1}{2}$| se obtiene $$\frac{1}{2}$$.
    Si yo deseo escribir una fracción en un párrafo, hago uso del comando \verb|\frac{num}{den}| en modo matemático, en ,donde se obtiene $\frac{1}{2}$, donde \verb|num| es el numerador y \verb|den| es el denominador, por ejemplo si escribimos \verb|$\frac{1}{2}$| se obtiene $\frac{1}{2}$.
    Si yo deseo escribir una fracción en un párrafo, hago uso del comando \verb|\frac{num}{den}| en modo matemático, en ,donde se obtiene $\frac{1}{2}$, donde \verb|num| es el numerador y \verb|den| es el denominador, por ejemplo si escribimos \verb|$$\dfrac{1}{2}$$| se obtiene $$\dfrac{1}{2}$$.
    Si yo deseo escribir una fracción en un párrafo, hago uso del comando \verb|\frac{num}{den}| en modo matemático, en ,donde se obtiene $\frac{1}{2}$, donde \verb|num| es el numerador y \verb|den| es el denominador, por ejemplo si escribimos \verb|$\frac{1}{2}$| se obtiene $\frac{1}{2}$.
    Si yo deseo escribir una fracción en un párrafo, hago uso del comando \verb|\frac{num}{den}| en modo matemático, en ,donde se obtiene $\frac{1}{2}$, donde \verb|num| es el numerador y \verb|den| es el denominador, por ejemplo si escribimos \verb|$\frac{1}{2}$| se obtiene $\frac{1}{2}$. Otra opción es la siguiente $$ \frac{1}{2} + \frac{3}{5}$$
    
   \section{Delimitadores}
      $$5(\frac{1}{2} + \frac{1}{3})$$
      
      $$5\left[\frac{1}{2} + \frac{1}{3}\right\}$$
   \section{Fracciones}
      $$ \dfrac{\frac{2}{5} + \frac{1}{3}}{\frac{1}{9}}$$

   \section{Proposiciones lógicas}
      $$ \neg p (\sim p), \quad p \vee q, \quad p \wedge q, \quad p \Rightarrow q, \quad p \Leftrightarrow q $$
    \section{Conjuntos}
      $$A = \left\{1,2,\frac{3}{2},4,5\right\}$$
    
      $$A = \left\{x \in \mathbb{N}\;|\; 1 \leq n \leq 5\right\}$$
    
      $$\N, \Z, \Q, \I, \R, \C$$
    
      $$A \cup B, \quad A \cap B, \quad A \subset B, a \in A, \quad a \notin A, \quad A \not\subset B,\quad \emptyset, \quad \varnothing, \quad A^c, \quad C_M(A), \quad A-B, \quad A \backslash B $$
      
     \section{Funciones}
       Sea la función $f: x \subset \R \to \R$, donde $y = f(x)$
       \subsection{Polinomial}
       $$f(x) = a_{n} x^{n} + a_{n-1} x^{n-1} + \ldots + a_{2} x^{2} + a_{1} x + a_{0}  $$
       
       $$f(x) = 3x^{3} + 5x^{2} - x + 1 $$
     \section{Exponencial}
       $$f(x) = a^{x}, \quad f(x) = e^{-x^{2}} = exp\left(-x^{2}\right)$$
     
     \section{Logaritmos}
       $$\log_{b}\left(x\right), \quad \log\left(x\right), \quad \ln\left(x\right) $$
     
     \section{Trigonométrica}
       $$\sen(x), \quad \cos(x), \quad \tan(x), \quad \cot(x), \quad \sec(x), \quad \csc(x), \quad \pi$$
     
     \section{Trigonométricas inversas}
      $$\arcsen(x), \quad \arccos(x), \quad \arctan(x), \quad \arccot(x), \quad \arcsec(x), \quad \arccsc(x)$$
     
     \section{Valor absoluto}
       $$\left|\frac{x}{2}\right|, \quad \left\vert\frac{x}{2}\right\vert$$
     
     \section{Máximo entero}
       $$[x], \quad \lfloor x\rfloor, \quad \ldbrack x\rdbrack $$
     
     \section{Signo}
       $$\text{sgn}(x) = \begin{cases}
                           1, & x > 0 \\
                           0, & x = 0\\
                           -1, & x < 0
                         \end{cases} = \left\{ \begin{array}{rc}
                                                    1, & x > 0 \\
                                                    0, & x = 0\\
                                                    -1, & x < 0 
                                             \end{array}\right.$$
     
     \section{Límites}
     $\lim_{x \to a} f(x)$ \lipsum[5]
     
       $$\left(\lim_{x \to a} f(x) = L\right) := \left( \forall \varepsilon  > 0, \exists\delta(\varepsilon) > 0, x \in \text{Dom}(f), 0 < | x - a| < \delta \Rightarrow |f(x) = L| < \varepsilon \right)$$
       
     \section{Derivadas}
     
         $$ f'(x), \frac{df}{d(x)}, \frac{d^2f}{dx^2}, \frac{\partial F}{\partial x}, \frac{\partial ^2F}{\partial x\partial y}, \dot{x}(t), \ddot{x}(t) $$
     
     \section{Integrales}
     
          $$  \int f(x)dx, \int_{0}^{1} f(x)dx, \iint_{D} f(u)du, \in_{a}^{b}\int_{c}^{d}f(x,y)dxdy, 
          \iiint_{D}f(v)dv, \oint_{L}f(s)ds $$
      
          
     \section{Suseciones}
     
        $$ f: \N \to X \subset \R, \qquad \left( x_{n} \right)_{n \in \N} \subset \R, 
        \qquad f(n) = x_{n} \qquad \text{Ran}(f) = \left\{x_{1},x_{2}, \ldots,x_{3},\ldots \right\}
        = \left\{x_{n}\right\}_{n \in \N}$$ 
      
      
     \section{Series}
      
        $$ \sum_{i=1}^{n}x_{i} , \sum_{i=1}^{\infty} x_{n} $$
      

     \section{Productoria}    
       
       $\prod_{i=1}^{n}x_{n}$
       
       $$ \prod_{i=1}^{n}x_{n} $$
         
         
     \section{Extras}
        
       $$ \Vert x \Vert, (x,y), \langle x,y \rangle, <x, y>, \bigcap_{i=1}^{n}A_{i}, \bigcup_{i=1}^{n}A_{i}, \mu, \sigma^{2}, \overline{X}, \overline{x}, S^{2} $$
    
     \section{Escritura de ecuaciones}
        
        \begin{equation}
            x^{2} + y^{2} = r^{2} \label{eq: circ}
        \end{equation}
       
        De la ecuación \eqref{eq: circ}
        
       \begin{align}
           f\left(\alpha_{1}x_{1} + \cdots + \alpha_{n}x_{n}\right) &= f\left(\alpha_{1}x_{1} +\beta \left( \frac{\alpha_{2}}{\beta}x_{2} + \cdots + \frac{\alpha_{n}}{\beta}x_{n}\right)\right) \leq  \notag\\
           & \leq \alpha_{1}f(x_{1}) + \beta f\left(\frac{\alpha_{2}}{\beta} + \cdots + \frac{\alpha_{n}}{\beta}x_{n}\right) \notag 
       \end{align}
    
       \begin{align*}
           f\left(\alpha_{1}x_{1} + \cdots + \alpha_{n}x_{n}\right) &= f\left(\alpha_{1}x_{1} +\beta \left( \frac{\alpha_{2}}{\beta}x_{2} + \cdots + \frac{\alpha_{n}}{\beta}x_{n}\right)\right) \leq  \\
           & \leq \alpha_{1}f(x_{1}) + \beta f\left(\frac{\alpha_{2}}{\beta} + \cdots + \frac{\alpha_{n}}{\beta}x_{n}\right) 
       \end{align*}
       
       
        \begin{equation}
            \begin{aligned}
                f\left(\alpha_{1}x_{1} + \cdots + \alpha_{n}x_{n}\right) &= f\left(\alpha_{1}x_{1} +\beta \left( \frac{\alpha_{2}}{\beta}x_{2} + \cdots + \frac{\alpha_{n}}{\beta}x_{n}\right)\right) \leq  \\
                & \leq \alpha_{1}f(x_{1}) + \beta f\left(\frac{\alpha_{2}}{\beta} + \cdots + \frac{\alpha_{n}}{\beta}x_{n}\right)     
            \end{aligned}
        \end{equation}
    
     
     \section{Entornos matemáticos}
      
        \begin{defi}[Sucesión]
            \lipsum[9] 
        \end{defi}
        
        
        \begin{lema}
          \lipsum[8]
        \end{lema}
        
        
        \begin{teo}
            \lipsum[7]
        \end{teo}


        \begin{proof}
          \lipsum[6]
        \end{proof}
     
     \section{Arreglos}
      
      $$ u = (2.3, -9.98, 8.1),\qquad  u' = \left(\begin{array}{r}
                                         2.30\\
                                        -9.98\\
                                         8.10
                                     \end{array} \right)$$
      
       
       \begin{equation*}
          A = \left( \begin{array}{rrr}
                    76 & 1 & 81\\
                    -8 & 20 & 1 \\
                     0 & -2 & 15
              \end{array} \right)
       \end{equation*}
       
\end{document}

