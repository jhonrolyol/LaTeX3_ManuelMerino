% Capítulo 2: Planteamiento del Problema
	\chapter{Planteamiento del Problema}\label{Cap: Planteamiento del Problema}
		\lipsum \lipsum[3]
		% Sección 2.1: Enunciado del Problema
			\section{Enunciado del Problema}\label{Sec: Enunciado del Problema}
				\lipsum[4]
				Como mensiona \cite{Koop-2003}, en su libro
				\lipsum[5]		
		% Sección 2.2: Formulación del Problema
			\section{Formulación del Problema}\label{Sec: Formulación del Problema}
				\lipsum[5]
				% Subsección 2.2.1: Problema General
					\subsection{Problema General}\label{Subsec: Problema General}	
						\lipsum[6]
				% Subsección 2.2.2: Problemas Específicos
					\subsection{Problemas Específicos}\label{Subsec: Problemas Específicos}
						\lipsum[1]
						Un texto es una composición de signos codificados en un sistema
						de escritura que forma una unidad de sentido. También es una composición
						de caracteres imprimibles generados por un algoritmo de cifrado que, aunque no
						tienen sentido para cualquier persona, sí puede ser descifrado por su destinatario original.
						\begin{enumerate}
							\item Un texto es una composición de signos codificados en un sistema
							de escritura que forma una unidad de sentido.
							\item Un texto es una composición de signos codificados en un sistema
							de escritura que forma una unidad de sentido.
							\item Un texto es una composición de signos codificados en un sistema
							de escritura que forma una unidad de sentido.
							\item Un texto es una composición de signos codificados en un sistema
							de escritura que forma una unidad de sentido.
						\end{enumerate}
		