% Capítulo 5: Marco Teórico
	\chapter{Marco Teórico}\label{Cap: Marco Teórico}
		Como se menciona en el trabajo de \cite{Johnson-1987}, la revisión de \lipsum[10]
		Consider the following first-order vector difference equation: 
		\begin{equation}\label{eq: MED}
			\left[
			\begin{array}{c}
				y_{t}  \\
				y_{t-1}\\
				y_{t-2}\\
				\vdots \\
				y_{t-p+1}
			\end{array}
			\right] =
			\left[
			\begin{array}{cccccc}
				\phi_{1} & \phi_{2} & \phi_{3} & \cdots & \phi_{p-1} & \phi_{p}\\
				1    &    0     &    0     & \cdots &      0     &     0   \\
				0    &    1     &    0     & \cdots &      0     &     0   \\
				\vdots  &  \vdots  &  \vdots  & \cdots &   \vdots   &  \vdots \\
				0    &    0     &    0     & \cdots &      1     &     0   \\
			\end{array}
			\right]
			\left[
			\begin{array}{c}
				y_{t-1}\\
				y_{t-2}\\
				y_{t-3}\\
				\vdots \\
				y_{t-p}
			\end{array}
			\right] + 
			\left[
			\begin{array}{c}
				w_{t}\\
				0    \\
				0    \\
				\vdots \\
				0
			\end{array}
			\right]
		\end{equation}
		% Sección 5.1: Marco Histórico
			\section{ Marco Histórico}\label{Sec: Marco Histórico}
				\lipsum[11].
				\begin{equation}\label{eq: LAG1}
					y_{t} = (a + bL)Lx_{t}
				\end{equation}
				is exactly the same as
				\begin{equation}\label{LAG2}
					y_{t} = (aL + bL^{2})x_{t} = ax_{t-1} + bx_{t-2}
				\end{equation}
				To take another example, 
				\begin{equation}\label{eq: LAG3}
					\begin{aligned}
						(1-\lambda_{1}L)(1-\lambda_{2}L)x_{t} &= 
						(1-\lambda_{1}L-\lambda_{2}L + \lambda_{1}\lambda_{2}L^{2})x_{t}\\
						&= 
						(1-\left[\lambda_{1}+\lambda_{2}\right]L + \lambda_{1}\lambda_{2}L^{2})x_{t}\\
						&= 
						x_{t}-(\lambda_{1} + \lambda_{2})x_{t-1} + (\lambda_{1}\lambda_{2})x_{t-2}
					\end{aligned}
				\end{equation}
		% Sección 5.2: Sistema Teórico
			\section{Sistema Teórico}\label{Sec: Sistema Teórico}
				\lipsum[12].
				Gráfica estadisticas:
				\begin{figure}[H]
					\centering
					\begin{tikzpicture}[scale = 1.2]
						\pie{10/{NSP}, 20/{Aprobados}, 30/{Retirados}, 35/{Jalados}, 5/{Susti con fe}}
					\end{tikzpicture}
					\caption{Resultados}
					\label{fig: Pie}
				\end{figure}
		% Sección 5.3: Marco Conceptual 
			\section{Marco Conceptual }\label{Sec: Marco Conceptual }
				\lipsum[13]
		% Sección 5.4: Marco Referencial
			\section{Marco Referencial}\label{Sec: Marco Referencial}
				\lipsum[14]	
					
	