% Annexo
	\annex 
	% Anexo 1
		\chapter{Anexo 1: Matriz de Consistencia}
			\begin{table}[h!]
				\centering
				\caption{Matriz de consistencia}
				\begin{tabular}{p{5cm}p{5cm}p{5cm}}
					\toprule
					Problema & Objetivo & Hipótesis\\
					\midrule
					\lipsum[2] & \lipsum[2] & \lipsum[2]\\
					\bottomrule
				\end{tabular}
				\vspace{2mm}
				\caption*{\it Nota: Elaboración propia.}
				\label{tab: Matriz de consistencia}
			\end{table}
	% Anexo 2
		\chapter{Anexo 2: Resultados}
			\begin{table}[h!]
				\centering
				\caption{Tabla de resultados}
				\begin{tabular}{p{5cm}p{5cm}p{5cm}}
					\toprule
					Resultado 1 & Resultado 2 & Resultado 3\\
					\midrule
					\lipsum[2] & \lipsum[2] & \lipsum[2]\\
					\bottomrule
				\end{tabular}
				\vspace{2mm}
				\caption*{\it Nota: Elaboración propia.}
				\label{tab: Resultados}
			\end{table}
	% Anexo 3
		\chapter{Anexo 3: Datos}
			\begin{table}[h!]
				\centering
				\caption{Datos}
				\begin{tabular}{p{5cm}p{5cm}p{5cm}}
					\toprule
					Variable 1 & Variable 2 & Varaible 3\\
					\midrule
					\lipsum[2] & \lipsum[2] & \lipsum[2]\\
					\bottomrule
				\end{tabular}
				\vspace{2mm}
				\caption*{\it Nota: Elaboración propia.}
				\label{tab: Datos}
			\end{table}
	% Anexo 4
		\chapter{Anexo 4: Código de Matlab}
			\lstinputlisting[language=Matlab]{Codigo/Matlab.m}
	% Anexo 4
		\chapter{Anexo 5: Código de Java}
			\lstinputlisting[language=Java]{Codigo/Java.java}
	% Anexo 4
		\chapter{Anexo 6: Código de Python}
			\lstinputlisting[language=Python]{Codigo/Python.py}
		
		
		